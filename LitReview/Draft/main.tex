\documentclass[a4]{article}
\usepackage[margin=1cm]{geometry}

\title{Literature Review Draft}
\author{Daniel Hess - 21971897}
\date{}

\begin{document}
\maketitle

\subsection*{Texture Analysis}
Because we're concerned with computer vision applications, we need to distinguish that we are talking about \textit{visual} texture and not \textit{tactile} texture. There are a few definitions but they all come to the same kind of understanding, the texture of an image (or region/section of an image) refers to the shape, colour and general uniformity (or lack thereof) of a particular region. The purpose then of texture analysis is characterising the features of these textures so that we can tackle computer vision problems.

A few of the problems we can look into solving with this field includes object recognition. In this case we're looking at characterising texture so we can find objects within an image. Sometimes the concern is the presence/absence of the object we're interested in and other times it's the position (or both). 



\end{document}